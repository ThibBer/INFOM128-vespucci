\subsection{Scénario d'évolution n°9}
\subsubsection{Scénario}
Ajouter une clef étrangère entre la colonne \mintinline{sql}{photo.dir} et \mintinline{sql}{directories.dir}

\subsubsection{Modifications}
Tous les schémas sont concernés par cette modification, il faut donc les éditer pour faire propager les modifications faites dans le code.

Etant donné qu'une nouvelle contrainte référentielle va être ajoutée, certaines lignes déjà existantes pourraient poser problème.
Pour qu'une contrainte de clef étrangère puisse exister, il est impératif que la colonne référencée ne contiennent que des valeurs uniques et qu'elles soient de même type.
Si besoin, certaines entrées de la table directories devront être supprimées pour respecter la contrainte d'unicité.

Pour l'instant, cette contrainte n'est pas totalement existante.
En effet, il existe déjà une relation entre ces 2 colonnes mais basée sur un \mintinline{sql}{LIKE}. Il n'y a donc aucune contrainte explicitement définie.


\begin{enumerate}
    \item Adapter la requête de création de la table
    \begin{itemize}
        \item \textbf{Fichier concerné} : \mintinline{sql}{PhotoIndex.java}
        \item \textbf{Emplacement} : \href{https://github.com/MarcusWolschon/osmeditor4android/blob/127fb689ad42c77558e4512e14de754e0561cd27/src/main/java/de/blau/android/photos/PhotoIndex.java#L88}{ligne 88}
        \item \textbf{Méthode} : \mintinline{java}{public synchronized void onCreate(SQLiteDatabase db)}
    \end{itemize}

    Pour créer une contrainte, il faut tout d'abord adapter la requête de création de la table en y ajoutant la contraite de référence.
    Les 2 colonnes ayant le même type, il n'y a pas d'autres modifications à apporter afin de pouvoir créer la clef étrangère.
    \begin{minted}{sql}
    CREATE TABLE IF NOT EXISTS photos (lat int, lon int, direction int DEFAULT NULL, dir VARCHAR, name VARCHAR, source VARCHAR DEFAULT NULL);
    \end{minted}

    doit être changée en y ajoutant la contrainte référentielle comme suit :
    \begin{minted}{sql}
    CREATE TABLE IF NOT EXISTS photos (lat int, lon int, direction int DEFAULT NULL, dir VARCHAR, name VARCHAR, source VARCHAR DEFAULT NULL) FOREIGN KEY(dir) REFERENCES directories(dir);
    \end{minted}

    \item Modifier la requête de sélection des photos liée aux directories.
    \begin{itemize}
        \item \textbf{Fichier concerné} : \mintinline{sql}{PhotoIndex.java}
        \item \textbf{Emplacement} : \href{https://github.com/MarcusWolschon/osmeditor4android/blob/127fb689ad42c77558e4512e14de754e0561cd27/src/main/java/de/blau/android/photos/PhotoIndex.java#L236}{ligne 236}
        \item \textbf{Méthode} : \mintinline{java}{private void indexDirectories()}
    \end{itemize}

    \begin{minted}{java}
    private void indexDirectories() {
        Log.d(DEBUG_TAG, "scanning directories");
        // determine at least a few of the possible mount points
        File sdcard = Environment.getExternalStorageDirectory(); // NOSONAR
        List<String> mountPoints = new ArrayList<>();
        mountPoints.add(sdcard.getAbsolutePath());
        mountPoints.add(sdcard.getAbsolutePath() + Paths.DIRECTORY_PATH_EXTERNAL_SD_CARD);
        mountPoints.add(Environment.getExternalStoragePublicDirectory(Environment.DIRECTORY_DOWNLOADS).getAbsolutePath()); // NOSONAR
        File storageDir = new File(Paths.DIRECTORY_PATH_STORAGE);
        File[] list = storageDir.listFiles();
        if (list != null) {
            for (File f : list) {
                if (f.exists() && f.isDirectory() && !sdcard.getAbsolutePath().equals(f.getAbsolutePath())) {
                    Log.d(DEBUG_TAG, "Adding mount point " + f.getAbsolutePath());
                    mountPoints.add(f.getAbsolutePath());
                }
            }
        }

        SQLiteDatabase db = null;
        Cursor dbresult = null;

        try {
            db = getWritableDatabase();
            dbresult = db.query(SOURCES_TABLE, new String[] { URI_COLUMN, LAST_SCAN_COLUMN, TAG_COLUMN }, TAG_COLUMN + " is NULL", null, null, null, null,
                    null);
            int dirCount = dbresult.getCount();
            dbresult.moveToFirst();
            // loop over the directories configured
            for (int i = 0; i < dirCount; i++) {
                String dir = dbresult.getString(0);
                long lastScan = dbresult.getLong(1);
                Log.d(DEBUG_TAG, dbresult.getString(0) + " " + dbresult.getLong(1));
                // loop over all possible mount points
                for (String m : mountPoints) {
                    File indir = new File(m, dir);
                    Log.d(DEBUG_TAG, "Scanning directory " + indir.getAbsolutePath());
                    if (indir.exists()) {
                        Cursor dbresult2 = null;
                        try {
                            dbresult2 = db.query(PHOTOS_TABLE, new String[] { "distinct dir" }, "dir LIKE '" + indir.getAbsolutePath() + "%'", null, null, null,
                                    null, null);
                            int dirCount2 = dbresult2.getCount();
                            dbresult2.moveToFirst();
                            for (int j = 0; j < dirCount2; j++) {
                                String photo = dbresult2.getString(0);
                                File pDir = new File(photo);
                                if (!pDir.exists()) {
                                    Log.d(DEBUG_TAG, "Deleting entries for gone photo " + photo);
                                    db.delete(PHOTOS_TABLE, URI_WHERE, new String[] { photo });
                                }
                                dbresult2.moveToNext();
                            }
                            dbresult2.close();
                            scanDir(db, indir.getAbsolutePath(), lastScan);
                            updateSources(db, indir.getName(), null, System.currentTimeMillis());
                        } finally {
                            close(dbresult2);
                        }
                        continue;
                    }
                    Log.d(DEBUG_TAG, "Directory " + indir.getAbsolutePath() + " doesn't exist");
                    // remove all entries for this directory
                    db.delete(PHOTOS_TABLE, URI_WHERE, new String[] { indir.getAbsolutePath() });
                    db.delete(PHOTOS_TABLE, "dir LIKE ?", new String[] { indir.getAbsolutePath() + "/%" });
                }
                dbresult.moveToNext();
            }
        } catch (SQLiteException ex) {
            // Don't crash just report
            ACRAHelper.nocrashReport(ex, ex.getMessage());
        } finally {
            close(dbresult);
            SavingHelper.close(db);
        }
    }
    \end{minted}


    Etant donné qu'une contrainte référentielle existe, il n'est plus nécessaire l'utiliser le mot clef \mintinline{sql}{LIKE} pour récupérer les photos qui appartient à un dossier mais un simple select suffit.

    \item Modifier la méthode d'insertion
    \begin{itemize}
        \item \textbf{Fichier concerné} : \mintinline{sql}{PhotoIndex.java}
        \item \textbf{Emplacement} : \href{https://github.com/MarcusWolschon/osmeditor4android/blob/127fb689ad42c77558e4512e14de754e0561cd27/src/main/java/de/blau/android/photos/PhotoIndex.java#L515}{ligne 515}
        \item \textbf{Méthode} : \mintinline{java}{private void insertPhoto(@NonNull SQLiteDatabase db, @NonNull Photo photo, @NonNull String name, @Nullable String source)}
    \end{itemize}

    \begin{minted}{java}
    private void insertPhoto(@NonNull SQLiteDatabase db, @NonNull Photo photo, @NonNull String name, @Nullable String source) {
        try {
            ContentValues values = new ContentValues();
            values.put(LAT_COLUMN, photo.getLat());
            values.put(LON_COLUMN, photo.getLon());
            if (photo.hasDirection()) {
                values.put(DIRECTION_COLUMN, photo.getDirection());
            }
            values.put(URI_COLUMN, photo.getRef());
            values.put(NAME_COLUMN, name);
            if (source != null) {
                values.put(SOURCE_COLUMN, source);
            }
            db.insert(PHOTOS_TABLE, null, values);
        } catch (SQLiteException sqex) {
            Log.d(DEBUG_TAG, sqex.toString());
            ACRAHelper.nocrashReport(sqex, sqex.getMessage());
        } catch (NumberFormatException bfex) {
            // ignore silently, broken pictures are not our business
        } catch (Exception ex) {
            Log.d(DEBUG_TAG, ex.toString());
            ACRAHelper.nocrashReport(ex, ex.getMessage());
        } // ignore
    }
    \end{minted}

    Cette dernière s'occupe d'insérer une photo. Dorénavent, il faut s'assurer que la propriété \mintinline{java}{dir} de la photo qu'on veut insérer existe dans la table \mintinline{java}{directories}.
    Une vérification avec un select permet de résoudre ce problème, sinon il est possible d'utiliser l'exception générée par la requete d'insertion.

    L'adaptation du code pourra donc se faire :
    \begin{itemize}
        \item dans la méthode ci-dessus, sans rien adapter. La méthode comporte déjà un \mintinline{java}{try/catch} qui empéchera à l'application de planter. Il serait toutefois intéressant de renvoyer l'information pour déterminer si l'insertion a eu lieu ou pas.
        \item dans la méthode ci-dessus, en y ajoutant un sélect afin de déterminer que le \mintinline{sql}{java} existe bien dans la table \mintinline{java}{directories}
        \item dans la méthode ci-dessus, en renvoyant une exception qui informe la méthode appellante qu'une erreur s'est produite. Il faudra également modifier l'appel à la méthode \mintinline{java}{insertPhoto} dans les autres classes qui l'utilise pour l'entourer d'un \mintinline{java}{try/catch} et ainsi éviter des crashs.
    \end

\end{enumerate}
