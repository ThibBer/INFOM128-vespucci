\subsection{Scénario d'évolution n°8}
\subsubsection{Scénario}
Suppression de la colonne description dans la table \mintinline{java}{presets}.

\subsubsection{Modifications}
Tous les schémas sont concernés par cette modification, il faut donc les éditer pour faire propager les modifications faites dans le code.

Cette modification entrainera la perte de donnée pour les presets. L'affichage devra également être adapté dans l’application.

Les modifications sont principalement centrées dans AdvancedPrefDatabase.

Tout d'abord, il faut supprimer la constante \mintinline{java}{DESCRIPTION_COL} à la ligne 74. Cette dernière correspond au nom de la colonne qu'on souhaite supprimer.

Grâce à cette constante, les IDE modernes aident grandement à trouver les endroits où la constante était utilisée. Dans le pire des cas, le compilateur prendra la relève et indiquera les endroits à modifier.

Voici toutefois les endroits importants à modifier.

\begin{enumerate}
    \item Dans la méthode getActivePresets, il faut supprimer la constante \mintinline{java}{DESCRIPTION_COL}
          \begin{itemize}
              \item \textbf{Fichier concerné} : \texttt{AdvancedPrefDatabase.java}
              \item \textbf{Emplacement} :
                    \href{https://github.com/MarcusWolschon/osmeditor4android/blob/dcabe8084aa15f5551a37c990516bf73398af1bf/src/main/java/de/blau/android/prefs/AdvancedPrefDatabase.java#L662C25-L662C41}{Ligne 662}
              \item \textbf{Ancien code} :
                    \begin{minted}{java}
                        @NonNull
                        public PresetInfo[] getActivePresets() {
                                SQLiteDatabase db = getReadableDatabase();
                                Cursor dbresult = db.query(PRESETS_TABLE,
                                new String[] { ID_COL, NAME_COL, VERSION_COL, SHORTDESCRIPTION_COL, DESCRIPTION_COL, URL_COL, LASTUPDATE_COL, ACTIVE_COL, USETRANSLATIONS_COL },
                                "active=1", null, null, null, POSITION_COL);
                                PresetInfo[] result = new PresetInfo[dbresult.getCount()];
                                Log.d(DEBUG_TAG, "#prefs " + result.length);
                                dbresult.moveToFirst();
                                for (int i = 0; i < result.length; i++) {
                                        Log.d(DEBUG_TAG, "Reading pref " + i + " " + dbresult.getString(1));
                                        result[i] = new PresetInfo(dbresult.getString(0), dbresult.getString(1), dbresult.getString(2), dbresult.getString(3), dbresult.getString(4),
                                        dbresult.getString(5), dbresult.getString(6), dbresult.getInt(7) == 1, dbresult.getInt(8) == 1);
                                        dbresult.moveToNext();
                                    }
                                dbresult.close();
                                db.close();
                                return result;
                            }
                    \end{minted}
              \item \textbf{Nouveau code} :
                    \begin{minted}{java}
                        @NonNull
                        public PresetInfo[] getActivePresets() {
                                SQLiteDatabase db = getReadableDatabase();
                                Cursor dbresult = db.query(PRESETS_TABLE,
                                new String[] { ID_COL, NAME_COL, VERSION_COL, SHORTDESCRIPTION_COL, URL_COL, LASTUPDATE_COL, ACTIVE_COL, USETRANSLATIONS_COL },
                                "active=1", null, null, null, POSITION_COL);
                                PresetInfo[] result = new PresetInfo[dbresult.getCount()];
                                Log.d(DEBUG_TAG, "#prefs " + result.length);
                                dbresult.moveToFirst();
                                for (int i = 0; i < result.length; i++) {
                                        Log.d(DEBUG_TAG, "Reading pref " + i + " " + dbresult.getString(1));
                                        result[i] = new PresetInfo(dbresult.getString(0), dbresult.getString(1), dbresult.getString(2), dbresult.getString(3), dbresult.getString(4),
                                        dbresult.getString(5), dbresult.getString(6), dbresult.getInt(7) == 1, dbresult.getInt(8) == 1);
                                        dbresult.moveToNext();
                                    }
                                dbresult.close();
                                db.close();
                                return result;
                            }
                    \end{minted}
          \end{itemize}
    \item Il faut également supprimer dbresult.getString(5) dans l'appel au constructeur de PresetInfo. De plus, il faut également décaler les index des autres appels à getString(val) car la supprimer d'une colonne sélectionnée décale tout le reste.
    \item La même modification que ci-dessus est à faire pour la méthode getPresets

    \item La méthode setPresetAdditionalFields est à adapter car elle utilise également la description lors de la modification d'un preset. Pour ce faire, il faut supprimer les lignes 776 à 778. De ce fait, la colonne \mintinline{java}{description} ne sera plus utilisée par le programme pour interagir avec la base de données.

    \item La dernière modification à effectuer est la suppression de l'attribut \mintinline{java}{description} de la classe PresetInfo.

    \item Les modifications engendrées par la suppression de cet attribut ne sont pas expliquées ici mais le compilateur permet de repérer les endroits où l'attribut \mintinline{java}{description} était utilisé. Il est donc aisé d'effectuer les derniers changements.
\end{enumerate}