\section{Conclusion}
Ce projet a permis d'approfondir les concepts de rétro-ingénierie appliqués à une base de données d'application Android, en l'occurrence Vespucci. En retraçant les schémas physique, logique et conceptuel, nous avons mis en lumière la structure sous-jacente de la base de données et identifié des opportunités d'amélioration.

L'étude des clés étrangères a révélé leur rôle central dans l'intégrité référentielle des données. Si certaines contraintes étaient implicites, notamment dans les relations entre les tables, le processus de rétro-ingénierie a permis de proposer des améliorations, telles que l'ajout explicite de clés étrangères. Ces ajustements ont renforcé la cohérence des données et simplifié les requêtes SQL nécessaires pour interagir avec elles.

En conclusion, cette analyse a mis en évidence l'importance des bonnes pratiques en conception de bases de données, notamment dans l'utilisation des clés étrangères pour établir des relations solides et explicites entre les entités. De plus, ce projet a souligné le rôle crucial de la documentation dans la compréhension et l'évolution des bases de données. Une documentation claire et détaillée facilite non seulement la rétro-ingénierie, mais aussi la maintenance et l'adaptabilité des systèmes face aux besoins changeants. Elle constitue un pilier fondamental pour assurer la pérennité et la cohérence des bases de données dans des environnements complexes et collaboratifs.