\section{Introduction}
Dans le cadre du cours INFOM218 à l'Université de Namur, ce rapport se concentre sur l'analyse et l'évolution des bases de données dans les applications Android, avec une étude de cas sur Vespucci, un éditeur OpenStreetMap open-source. Ce projet vise à effectuer une ingénierie inverse de la base de données, à élaborer des scénarios d'évolution, et à proposer des modifications pour améliorer la structure et la performance de l'application. En exploitant des outils tels que SQLInspect et des scripts personnalisés, nous avons étudié le schéma physique, logique et conceptuel de la base de données, tout en identifiant des opportunités d'optimisation. Ce rapport détaille nos démarches méthodologiques, nos découvertes et les solutions proposées pour adapter la base de données aux besoins évolutifs de l'application.
Afin de garantir la transparence sur la répartition des tâches au sein du groupe et de permettre une consultation des travaux réalisés, le code produit durant ce projet a été publié sur un dépôt GitHub accessible via le lien suivant : \href{https://github.com/ThibBer/INFOM128-vespucci}{INFOM128-vespucci}.