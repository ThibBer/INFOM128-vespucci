\subsection{Scénario d'évolution n°4}
\subsubsection{Scénario}
Fusionner deux tables en une seule: fusionner les tables \mintinline{java}{filterentries} et \mintinline{java}{filters}.

\textbf{Raison éventuelle}: Ces tables sont liées par une clé étrangère (\mintinline{java}{filter}) et décrivent toutes les deux les filtres et leurs entrées.

Les tables seront fusionnées en une table commune : filters.
\subsubsection{Modifications}
Tous les schémas sont concernés par cette modification, il faut donc les éditer pour faire propager les modifications faites dans le code.

\begin{enumerate}
    \item \textbf{Modifier la méthode \mintinline{java}{onCreate} qui crée les deux tables}
          \begin{itemize}
              \item \textbf{Fichier concerné} : \texttt{TagFilterDatabaseHelper.java}
              \item \textbf{Emplacement} :
                    \href{https://github.com/MarcusWolschon/osmeditor4android/blob/dcabe8084aa15f5551a37c990516bf73398af1bf/src/main/java/de/blau/android/filter/TagFilterDatabaseHelper.java#L28}{Ligne 28}
              \item \textbf{Ancien code} :
                    \begin{minted}{java}
                        @Override
                        public void onCreate(SQLiteDatabase db) {
                            try {
                                db.execSQL("CREATE TABLE filters (name TEXT)");
                                db.execSQL("INSERT INTO filters VALUES ('Default')");
                                db.execSQL(
                                        "CREATE TABLE filterentries (filter TEXT, include INTEGER DEFAULT 0, type TEXT DEFAULT '*', key TEXT DEFAULT '*', value TEXT DEFAULT '*', active INTEGER DEFAULT 0, FOREIGN KEY(filter) REFERENCES filters(name))");
                            } catch (SQLException e) {
                                Log.w(DEBUG_TAG, "Problem creating database", e);
                            }
                        }
                    \end{minted}
              \item \textbf{Nouveau code} :
                    \begin{minted}{java}
                        @Override
                        public void onCreate(SQLiteDatabase db) {
                            try {
                                db.execSQL(
                                        "CREATE TABLE filters (" +
                                        "id INTEGER PRIMARY KEY AUTOINCREMENT, " + // Identifiant unique pour chaque entrée
                                        "name TEXT NOT NULL, " +                  // Nom du filtre
                                        "include INTEGER DEFAULT 0, " +           // Inclusion
                                        "type TEXT DEFAULT '*', " +               // Type
                                        "key TEXT DEFAULT '*', " +                // Clé
                                        "value TEXT DEFAULT '*', " +              // Valeur
                                        "active INTEGER DEFAULT 0" +              // Indicateur actif
                                        ")");
                                // Insérer le filtre par défaut
                                db.execSQL("INSERT INTO filters (name, include, type, key, value, active) VALUES ('Default', 0, '*', '*', '*', 0)");
                            } catch (SQLException e) {
                                Log.w(DEBUG_TAG, "Problem creating database", e);
                            }
                        }
                    \end{minted}
          \end{itemize}
    \item \textbf{Adapter la query suivante}
          \begin{itemize}
              \item \textbf{Fichier concerné} : \texttt{TagFilterDatabaseHelper.java}
              \item \textbf{Emplacement} :
                    \href{https://github.com/MarcusWolschon/osmeditor4android/blob/dcabe8084aa15f5551a37c990516bf73398af1bf/src/main/java/de/blau/android/filter/TagFilter.java#L145}{Ligne 145}
              \item \textbf{Ancien code} :
                    \begin{minted}{java}
                        @Override
                        public void init(Context context) {
                            try (TagFilterDatabaseHelper tfDb = new TagFilterDatabaseHelper(context); SQLiteDatabase mDatabase = tfDb.getReadableDatabase()) {
                                //
                                filter.clear();
                                Cursor dbresult = mDatabase.query("filterentries", new String[] { "include", "type", "key", "value", "active" }, "filter = ?",
                                        new String[] { DEFAULT_FILTER }, null, null, null);
                                dbresult.moveToFirst();
                                for (int i = 0; i < dbresult.getCount(); i++) {
                                    try {
                                        filter.add(new FilterEntry(dbresult.getInt(0) == 1, dbresult.getString(1), dbresult.getString(2), dbresult.getString(3),
                                                dbresult.getInt(4) == 1));
                                    } catch (PatternSyntaxException psex) {
                                        Log.e(DEBUG_TAG, "exception getting FilterEntry " + psex.getMessage());
                                        if (context instanceof Activity) {
                                            ScreenMessage.barError((Activity) context,
                                                    context.getString(R.string.toast_invalid_filter_regexp, dbresult.getString(2), dbresult.getString(3)));
                                        }
                                    }
                                    dbresult.moveToNext();
                                }
                                dbresult.close();
                            }
                        }
                    \end{minted}
              \item \textbf{Nouveau code} :
                    \begin{minted}{java}
                        Cursor dbresult = mDatabase.query("filters", new String[] { "include", "type", "key", "value", "active" }, "name = ?", new String[] { DEFAULT_FILTER }, null, null, null);
                    \end{minted}
          \end{itemize}
\end{enumerate}
% TODO : Explain modifications for other queries