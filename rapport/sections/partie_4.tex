\section{Partie 4}

\subsection{Schéma de la base de données}

\paragraph{Mérites}
\begin{itemize}
    \item \textbf{Nommage explicite :} Les tables et les colonnes ont des noms descriptifs, donc on sait directement de quoi il s'agit.
    \item \textbf{Utilisation de clés étrangères :} Certaines relations entre les tables utilisent des clés étrangères, assurant l'intégrité référentielle (par exemple, entre les tables \mintinline{java}{filters} et \mintinline{java}{filterentries}).
    \item \textbf{Segmentation claire des données :} Les tables sont divisées correctement en fonction de leurs fonctionnalités (filtres, photos, headers, etc.).
\end{itemize}

\paragraph{Inconvénients}
\begin{enumerate}
    \item \textbf{Tables et colonnes non utilisées :}
          \begin{itemize}
              \item La table \mintinline{java}{t_renderer} et sa colonne \mintinline{java}{id} ne sont pas utilisées.
          \end{itemize}
    \item \textbf{Colonnes de clés étrangères nullables :}
          \begin{itemize}
              \item Plusieurs champs de clés étrangères autorisent les valeurs nulles, ce qui peut entraîner des problèmes d'intégrité des données.
          \end{itemize}
    \item \textbf{Relations implicites :}
          \begin{itemize}
              \item Certaines relations sont implicites, comme le lien entre \mintinline{java}{photos} et \mintinline{java}{directories} via une requête \mintinline{java}{LIKE} plutôt qu'une clé étrangère définie.
          \end{itemize}
    \item \textbf{Données dénormalisées :}
          \begin{itemize}
              \item Des tables surchargées comme \mintinline{java}{photos} stockent à la fois des métadonnées et des informations géospatiales, réduisant la clarté et la normalisation.
          \end{itemize}
    \item \textbf{Conception des clés primaires :}
          \begin{itemize}
              \item La clé primaire pour \mintinline{java}{tiles} est composite, ce qui peut compliquer l'indexation et les requêtes.
          \end{itemize}
\end{enumerate}

\paragraph{Recommandations}
\begin{itemize}
    \item Supprimer les tables et colonnes inutilisées, comme \mintinline{java}{t_renderer}, pour réduire la complexité du schéma et améliorer la maintenabilité.
    \item Définir des clés étrangères explicites pour les relations implicites, comme \mintinline{java}{photos.dir} et \mintinline{java}{directories.dir}, afin d'assurer la cohérence des données.
    \item Normaliser les données en scindant les tables surchargées (par exemple, diviser \mintinline{java}{photos} en \mintinline{java}{photo_metadata} et \mintinline{java}{photo_location}).
    \item S'assurer que toutes les clés étrangères font référence à des champs uniques et non nuls pour éviter les problèmes d'intégrité.
    \item Réévaluer l'utilisation des clés primaires composites et envisager d'introduire des clés substitutives là où c'est pertinent.
\end{itemize}

\subsection{Code de manipulation de la base de données}

\paragraph{Avantages}
\begin{itemize}
    \item \textbf{Gestion centralisée des requêtes SQL :} Les requêtes sont bien regroupées dans des fichiers Java spécifiques, facilitant la navigation et les mises à jour du code.
    \item \textbf{Requêtes lisibles :} La plupart des requêtes SQL sont simples et directes, facilitant leur compréhension.
\end{itemize}

\paragraph{Inconvénients}
\begin{enumerate}
    \item \textbf{Requêtes codées en dur :}
          \begin{itemize}
              \item Les requêtes sont souvent codées en dur sous forme de chaînes, ce qui peut poser des défis de maintenance et des risques d'injection SQL.
          \end{itemize}
    \item \textbf{Utilisation limitée des requêtes paramétrées :}
          \begin{itemize}
              \item Certaines requêtes utilisent des chaînes concaténées au lieu de la paramétrisation, augmentant les risques pour la sécurité.
          \end{itemize}
    \item \textbf{Absence de tests unitaires pour les requêtes :}
          \begin{itemize}
              \item Le manque de tests automatisés pour les instructions SQL peut entraîner des bugs non détectés pendant le développement.
          \end{itemize}
    \item \textbf{Évolution manuelle du schéma :}
          \begin{itemize}
              \item Les mises à jour du schéma nécessitent des modifications manuelles des requêtes, augmentant les risques d'erreurs pendant l'évolution du schéma.
          \end{itemize}
\end{enumerate}

\paragraph{Recommandations}
\begin{itemize}
    \item Passer à des requêtes paramétrées ou à un framework ORM (par exemple, \texttt{Room} pour Android) pour améliorer la sécurité et la maintenabilité.
    \item Développer une suite de tests complète pour les opérations sur la base de données afin d'assurer leur exactitude.
    \item Introduire des scripts de migration de schéma ou des bibliothèques comme \texttt{Flyway} ou le système de migration intégré de \texttt{Room} pour gérer les changements de schéma de manière systématique.
    \item Consolider la logique des requêtes SQL en utilisant des classes ou méthodes utilitaires pour réduire la duplication et centraliser les modifications.
\end{itemize}
